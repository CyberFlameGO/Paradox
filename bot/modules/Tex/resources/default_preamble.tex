% Always typeset math in display style
\everymath{\displaystyle}

% Use a larger font size
\usepackage[fontsize=14pt]{scrextend}

% Standard mathematical typesetting packages
\usepackage[T1]{fontenc}
\usepackage{amsthm, amsmath, amssymb}
\usepackage[fixamsmath]{mathtools}  % Extension to amsmath
\usepackage[cal=cm, scr=rsfs, frak=euler, bb=ams]{mathalpha}
\usepackage{lmodern}

% Symbol and utility packages
\usepackage{cancel}
\usepackage[nointegrals]{wasysym}

% Extras
\usepackage{physics}  % Lots of useful shortcuts and macros
\usepackage{quiver}  % For drawing commutative diagrams easily
\usepackage{hwemoji}  % Support Unicode emojis
\DeclareUnicodeCharacter{FE0E}{}
\DeclareUnicodeCharacter{FE0F}{}

% Common shortcuts
\def\mbb#1{\mathbb{#1}}
\def\mfk#1{\mathfrak{#1}}
\def\bN{\mbb{N}}
\def\bC{\mbb{C}}
\def\bR{\mbb{R}}
\def\bQ{\mbb{Q}}
\def\bZ{\mbb{Z}}

% Sometimes helpful macros
\DeclarePairedDelimiter\floor{\lfloor}{\rfloor}
\DeclarePairedDelimiter\ceil{\lceil}{\rceil}

% Some standard theorem definitions
\newtheorem{Theorem}{Theorem}
\newtheorem{Proposition}{Theorem}
\newtheorem{Lemma}[Theorem]{Lemma}
\newtheorem{Corollary}[Theorem]{Corollary}

\theoremstyle{definition}
\newtheorem{Definition}[Theorem]{Definition}

\renewcommand{\div}{\divisionsymbol}